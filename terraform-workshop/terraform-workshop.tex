
\documentclass[9pt]{beamer}

\usepackage[latin1]{inputenc}
\usepackage{colortbl}
\usepackage[english]{babel}

\newcommand{\myblue} [1] {{\color{blue}#1}}

% beamer template
\beamertemplatetransparentcovereddynamic
\usetheme{default}
\date{}

\title[Terraform workshop]{Terraform workshop}
\author[Saurabh Hirani]{Saurabh Hirani}
\institute[Autodesk]
{
\medskip
\myblue{\href{mailto:saurabh.hirani@gmail.com}{\texttt{saurabh.hirani@gmail.com}}} \\
}

\begin{document}
\setbeamercovered{invisible}

\frame{\titlepage
}

\part<presentation>{Main Talk}

\section[slides]{slides}

\begin{frame}[fragile]
  \frametitle{What this session is:}

  \begin{itemize}[<+->]
    \item Why Terraform?
    \item Iterative hands on
  \end{itemize}

\end{frame}


\begin{frame}[fragile]
  \frametitle{What this session is not:}

  \begin{itemize}
    \item AWS training
    \item Terraform reference
  \end{itemize}

\end{frame}

\begin{frame}[fragile]
  \frametitle{I can learn it on my own}

  \begin{itemize}
    \item \myblue{\href{https://www.terraformupandrunning.com/}{\texttt{Read this book}}}
    \item \myblue{\href{https://blog.gruntwork.io/an-introduction-to-terraform-f17df9c6d180}{\texttt{Read this blog}}}
    \item \myblue{\href{https://terraform.io/}{\texttt{Refer this site}}}
  \end{itemize}
\end{frame}

\begin{frame}[fragile]
  \frametitle{Why attend this session?}

  \begin{itemize}[<+->]
    \item No one goes from a messed up working setup to a clean working setup
    \item We will do that
  \end{itemize}

\end{frame}

\begin{frame}[fragile]
  \frametitle{Why Terraform?}

  \begin{itemize}
    \item Short answer - it is useful, battle tested and has a strong community
  \end{itemize}
\end{frame}


\begin{frame}[fragile]
  \frametitle{Why manual infra creation is a bad idea?}

  \begin{itemize}[<+->]
    \item Manual $\Rightarrow$ Error prone
    \item Parity loss between dev, stage, prod
    \item Boring!
  \end{itemize}

\end{frame}

\begin{frame}[fragile]
  \frametitle{There has to be a better way}

  \begin{itemize}[<+->]
    \item aws cli shell scripts?
    \item For python users \myblue{\href{https://github.com/boto/boto}{\texttt{boto}}}
    \item For ruby users \myblue{\href{https://github.com/fog/fog}{\texttt{fog}}}
  \end{itemize}

\end{frame}

\begin{frame}[fragile]
  \frametitle{Advantages of homegrown scripts}

  \begin{itemize}[<+->]
    \item Customized for your product
    \item You wrote it - you know all about it
  \end{itemize}

\end{frame}

\begin{frame}[fragile]
  \frametitle{Disadvantages of homegrown scripts}

  \begin{itemize}[<+->]
    \item Customized for your product
    \item You wrote it - \textbf{only} you know all about it
    \item Need a central box to manage
    \item State information?
    \item Audit trail?
    \item Is this what you are supposed to do?
    \item New product == new infra scripts?
  \end{itemize}

\end{frame}

\begin{frame}[fragile]
  \frametitle{We need a tool that...}

  \begin{itemize}[<+->]
    \item Helps automate infra creation
    \item Can be used by multiple users
    \item Promotes reusability
    \item Is not our headache to maintain
  \end{itemize}

\end{frame}

\begin{frame}[fragile]
  \frametitle{Enter Terraform}
\end{frame}

\begin{frame}[fragile]
  \frametitle{Features}

  \begin{itemize}[<+->]
    \item Infrastructure as code
    \item Ease of learning curve + active community
    \item Support for multiple cloud vendors
    \item State management
    \item Modules!
  \end{itemize}

\end{frame}

\begin{frame}[fragile]
  \frametitle{Comparison with Cloudformation}

  \begin{itemize}[<+->]
    \item Biggest advantage of TF over CF - TF is cloud vendor agnostic
    \item CF verbosity $>$ TF
    \item CF community $<$ TF
    \item CF learning curve =\~{} TF
    \item Other alternatives:
      \begin{itemize}[<+->]
        \item \myblue{\href{http://docs.ansible.com/ansible/latest/modules/list_of_cloud_modules.html}{\texttt{Ansible cloud modules}}}
        \item \myblue{\href{https://www.sparkleformation.io/}{\texttt{Sparkleformation}}}
        \item Many others
      \end{itemize}
  \end{itemize}
\end{frame}


\begin{frame}[fragile]
  \frametitle{Choose your editor}
  \begin{itemize}
    \item \myblue{\href{https://atom.io/packages/language-terraform}{\texttt{Atom syntax support}}}
    \item \myblue{\href{https://atom.io/packages/linter-terraform-syntax}{\texttt{Atom linting support}}}
    \item \myblue{\href{https://github.com/hashivim/vim-terraform}{\texttt{Vim syntax support}}}
    \item \myblue{\href{https://github.com/juliosueiras/vim-terraform-completion}{\texttt{Vim syntax support}}}
    \item Choose your weapon - syntax + basic linting (demo)
    \item Use "terraform validate" otherwise
  \end{itemize}

\end{frame}


\begin{frame}[fragile]
  \frametitle{Setup}
  \begin{itemize}
    \item 1 VPC
    \item 1 public subnet, 1 private subnet
    \item 1 ELB with public subnet + public security group
    \item 2 instances behind ELB in private subnet + private security group
    \item Hard to actually do the entire setup - simulate via s3
  \end{itemize}

\end{frame}

\begin{frame}[fragile]
  \frametitle{Demo}
  \begin{itemize}
    \item \myblue{\href{https://github.com/saurabh-hirani/terraform-workshop}{\texttt{Clone terraform-workshop repo}}}
  \end{itemize}

\end{frame}

\begin{frame}[fragile]
  \frametitle{assignment-1}

  \begin{itemize}[<+->]
    \item One file to rule them all - main.tf
    \item Good: works
    \item Bad: Hard to maintain
  \end{itemize}

\end{frame}

\begin{frame}[fragile]
  \frametitle{assignment-2}

  \begin{itemize}[<+->]
    \item Split main.tf
    \item Good: Easier to read
    \item Bad: Does not handle different environments
    \item Worth mentioning: \myblue{\href{https://www.terraform.io/docs/state/workspaces.html}{\texttt{terraform workspaces}}}
  \end{itemize}

\end{frame}

\begin{frame}[fragile]
  \frametitle{assignment-3}

  \begin{itemize}[<+->]
    \item Split same code across multiple environments
    \item Good: works
    \item Bad: Repetition of code - only env different
  \end{itemize}

\end{frame}


\begin{frame}[fragile]
  \frametitle{assignment-4}

  \begin{itemize}[<+->]
    \item tfvars to abstract out common code
    \item Good: works
    \item Bad: same as previous case but now both dev stage exactly the same
  \end{itemize}

\end{frame}


\begin{frame}[fragile]
  \frametitle{assignment-5}

  \begin{itemize}[<+->]
    \item local module
    \item Good: works
    \item Bad: Infra and module tightly coupled
  \end{itemize}

\end{frame}


\begin{frame}[fragile]
  \frametitle{assignment-6}

  \begin{itemize}[<+->]
    \item remote module
    \item Good: decoupling, versioning
    \item Bad: Long terraform commands to document
  \end{itemize}

\end{frame}


\begin{frame}[fragile]
  \frametitle{Modules}

  \begin{itemize}[<+->]
    \item Rakefile for release versioning
    \item terraform-docs for auto generating documentation
  \end{itemize}

\end{frame}

\begin{frame}[fragile]
  \frametitle{assignment-7}

  \begin{itemize}[<+->]
    \item add Makefile to the mix
    \item Good: make X helps
    \item Bad: create/destroy everything together - no staged approach
  \end{itemize}

\end{frame}


\begin{frame}[fragile]
  \frametitle{assignment-8}

  \begin{itemize}[<+->]
    \item Split the creation keeping in mind the infra and/or the users
    \item Remote state
    \item Good: cleaner, modular than previous approach
    \item Bad: Don't know the dependencies by looking at the structure
  \end{itemize}

\end{frame}


\begin{frame}[fragile]
  \frametitle{Splitting infra creation advantages}

  \begin{itemize}[<+->]
    \item Allows closer inspection
    \item Plan with local source, apply in dev with git source
    \item \myblue{\href{https://github.com/saurabh-hirani/terraform-workshop/blob/master/assignment-8/dev/base/main.tf\#L2}{\texttt{Example}}}
    \item Remote state allows team to collaborate
  \end{itemize}

\end{frame}

\begin{frame}[fragile]
  \frametitle{Splitting infra creation advantages}

  \begin{itemize}[<+->]
    \item Allows closer inspection
    \item Plan locally, apply in dev via versioning and take it from there
    \item Remote state allows team to collaborate
  \end{itemize}

\end{frame}

\begin{frame}[fragile]
  \frametitle{Naming: to delegate or not to delegate?}
  \begin{itemize}[<+->]
    \item Approach 1: pass primitives, module constructs the name
    \item \myblue{\href{https://github.com/saurabh-hirani/terraform-workshop-module/blob/master/vpc/main.tf\#L1\#L11}{\texttt{Example}}}
      \begin{itemize}[<+->]
        \item More Uniform
        \item Less flexible
        \item Preferred for inner source (think moniker)
      \end{itemize}
    \item Approach 2: construct names locally and pass to the module
    \item \myblue{\href{https://github.com/saurabh-hirani/terraform-workshop-module/blob/master/vpc/main.tf\#L13\#L23}{\texttt{Example}}}
      \begin{itemize}[<+->]
        \item Less Uniform
        \item More flexible
        \item Preferred for open source
      \end{itemize}
  \end{itemize}
\end{frame}


\begin{frame}[fragile]
  \frametitle{One shot v/s incremental}
  \begin{itemize}[<+->]
    \item Avoid extremeties: whole universe v/s each component
    \item Do simple splits: vpc, base, app
    \item One shot:
      \begin{itemize}
        \item Less repetition
        \item Magical?
        \item \myblue{\href{https://www.terraform.io/docs/commands/destroy.html}{\texttt{Manually do targeted destroys}}}
      \end{itemize}
    \item Incremental:
      \begin{itemize}
        \item More repetition
        \item Remember dependency order
        \item One step at a time
      \end{itemize}
  \end{itemize}
\end{frame}

\begin{frame}[fragile]
  \frametitle{assignment-9}

  \begin{itemize}[<+->]
    \item Use simple numbering to define steps
    \item Hack
    \item Needs a better way to manage folder level dependencies
  \end{itemize}

\end{frame}


\begin{frame}[fragile]
  \frametitle{More material}

  \begin{itemize}
    \item \myblue{\href{https://blog.gruntwork.io/an-introduction-to-terraform-f17df9c6d180}{\texttt{Yevgeniy's awesome terraform tutorial}}}
    \item \myblue{\href{https://registry.terraform.io/}{\texttt{Terraform module registry}}}
    \item \myblue{\href{https://www.terraform.io/docs/configuration/interpolation.html}{\texttt{terraform interpolation syntax}}}
    \item \myblue{\href{https://github.com/kamatama41/tfenv}{\texttt{tfenv}}}
  \end{itemize}

\end{frame}

\begin{frame}[fragile]
  \frametitle{Q \& A}
\end{frame}

\end{document}
